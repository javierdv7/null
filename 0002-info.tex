\documentclass[12pt]{article}
\usepackage[spanish]{babel}
\usepackage[utf8]{inputenc}
\usepackage[T1]{fontenc}
\usepackage{geometry}
\geometry{margin=2.5cm}
\usepackage{lmodern}
\usepackage{hyperref}
\hypersetup{colorlinks=true, linkcolor=black, urlcolor=blue}

\title{Informe Técnico}
\author{Javier David Vargas Torres}
\date{12 de mayo de 2025}

\begin{document}

\maketitle

\newpage
\section*{Origen del Código}

El algoritmo fue recuperado desde una unidad USB sin etiqueta, hallada detrás de una terminal abandonada en el subnivel -1 de la Escuela de Ingeniería. El contenido inicial era ilegible, compuesto por secuencias binarias sin formato claro. Tras aplicar múltiples decodificaciones y filtros de conversión, emergió un ejecutable funcional. Su comportamiento, aunque aparentemente programático, revela patrones inquietantemente intencionales.

\section*{Autoría y Responsabilidad}

Soy Javier David Vargas Torres. Reconozco haber interpretado y ejecutado el contenido de este algoritmo en un entorno aislado (hasta donde tengo certeza, sigue siéndolo). No reclamo su autoría, pero asumo la responsabilidad de haberlo liberado de su estado latente.

\section*{Descripción Funcional del Algoritmo}

El código simula la secuencia de arranque de un sistema operativo ficticio. Durante su ejecución:
\begin{itemize}
    \item Conecta múltiples capas de información en paralelo.
    \item Realiza un escaneo exhaustivo del entorno físico y lógico.
    \item Emite señales auditivas tipo "beep" con envolvente sonora.
    \item Muestra datos del sistema como si provinieran de una entidad autoconsciente.
\end{itemize}

El comportamiento observado sugiere un propósito más narrativo que práctico, aunque su estructura permite interpretarlo como un simulador de introspección digital.

\section*{Notas Adicionales}

No existen pruebas concluyentes que vinculen este algoritmo con entidades externas o redes activas. Sin embargo, múltiples usuarios han reportado una sensación de “ser observados” durante su ejecución. A la fecha, ninguna anomalía física ha sido atribuida a su uso.

\vspace{1cm}
\begin{center}
\textit{Este informe se entrega como testimonio técnico. Su interpretación queda bajo la discreción del lector.}
\end{center}

\end{document}
